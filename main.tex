\documentclass[12pt]{article}

\usepackage[T2A]{fontenc}
\usepackage[utf8]{inputenc}
\usepackage[russian]{babel}
\usepackage{latexsym,amsfonts,amssymb,amsthm,amsmath}
\usepackage{tikz}
\usepackage{wrapfig}
\usepackage{physics}
\usepackage{enumerate}
\usepackage{indentfirst}
\usepackage{array}
\usepackage{url}
\usepackage{ulem}

\setlength{\oddsidemargin}{0in}
\setlength{\textwidth}{6.5in}
\setlength{\textheight}{8.8in}
\setlength{\topmargin}{0in}
\setlength{\headheight}{18pt}

\newcommand{\R}{\mathbb{R}}
\newcommand{\N}{\mathbb{N}}
\newcommand{\Z}{\mathbb{Z}}
\newcommand{\Q}{\mathbb{Q}}
\newcommand{\cov}{\mathop{\rm cov}}

\newenvironment{task}[1]{
    \subsection*{Задача #1}
}{
    \vspace{0.2in}
}

\title{Машинное обучение --- 2 \\
        \large Ядровые методы}

\author{Полина Гусева, Анастасия Дроздова, Дарья Сапожникова \\
        \large БПМИ181}
\date{}

\begin{document}

\maketitle

\vspace{0.5in}

\par Ядровые методы --- класс алгоритмов машинного обучения, использующих ядровую функцию. За счет ядровой функции методы неявным образом оперируют объектами в многомерном, возможно бесконечномерном, пространстве. Большинство методов из этого класса хорошо статистически обоснованы.
\par Ядровые методы получили широкое распространение в конце 1990-х, когда было показано, что в задаче распознавания изображений из набора данных MNIST ядровый SVM показывает результаты, аналогичные нейронным сетям того времени \cite{mnistwebsite}.

\section*{Теория ядровых методов}
\par Как мы уже знаем из линейной алгебры, \textit{евклидово пространство} --- это конечномерное вещественное векторное пространство с введенным на нем скалярным произведением. 
\par Скалярное произведение $ \langle \cdot, \cdot \rangle $ порождает на пространстве метрику:
\[
    \rho(x, y) = \sqrt{\langle x - y, x - y \rangle}
\]
\par Таким образом, евклидово пространство является метрическим. Будем называть метрическое пространство \textit{полным}, если каждая фундаментальная последовательность на нем сходится к элементу этого пространства. Можно показать, что любое евклидово пространство является полным \cite{rudineuclideanproof}.
\par \textit{Гильбертово пространство} является обобщением евклидова пространства. Гильбертово пространство допускает бесконечномерность, но при этом требуется, чтобы пространство обязательно было полным. Полнота пространства гарантирует существование ортонормированного базиса, за счет чего с гильбертовыми пространствами часто можно работать по аналогии с конечномерными пространствами \cite{useofcompleteness}.

\par В дальнейшем будем рассматривать гильбертово пространство на функциях.

\par Будем говорить, что $ K : X \times X \to \R $ --- \textit{воспрозводящее ядро} гильбертова пространства $ \mathbb{H} $, если выполняются следующие условия:
\begin{itemize}
    \item $ \forall x \in X $ верно, что $ K_x(z) = K(x, z) \in \mathbb{H} $
    \item $ \forall x \in X, \forall f \in \mathbb{H} $ верно $ f(x) = \langle f, K_x \rangle $
\end{itemize}
\par Для любого ядра существует гильбертово пространство, для которого оно является воспроизводящим.

\par Примечательно, что в гильбертовых пространствах с вопроизводящим ядром оптимизационные задачи имеют решение в довольно простой форме. Следующая теорема будет именно об этом.
\par \textit{Теорема о представителе\footnote{Про эту теорему я нашла только тексты на английском, где она называется Representer theorem. Возможно, она все же как-то называется на русском, но гугл мне не раскрыл эту тайну.}.} Пусть $ \mathbb{H} $ --- гильбертово пространство с воспроизводящим ядром $ K $ и нормой $ \norm{h} = \sqrt{\langle h, h \rangle} $. Пусть $ G : \R \to \R $ --- монотонно возрастающая функция, а $ L $ --- функция потерь. Тогда любое решение оптимизационной задачи
\[
    \arg \min_{h \in \mathbb{H}} \left(G(\norm{h}) + L(h(x_1), \ldots, h(x_\ell))\right)
\]
\par имеет вид
\[
    x \mapsto \alpha_1 K(x, x_1) + \ldots + \alpha_\ell K(x, x_\ell)
\]
\par Доказательство теоремы \sout{оставим любопытному читателю} можно найти в Википедии.

\par Все это говорит о том, что ядровый метод по сути подбирает в соответствующем гильбертовом пространстве функцию, при помощи которой отражает объекты в спрямляющее пространство.

\section*{Приложения}
\par У ядровых методов есть множество преимуществ. Помимо очевидных плюсов, таких как нелинейная разделяющая поверхность и эффективная работа со скалярными произведениями в многомерных пространствах, есть еще возможность работать с данными, которые не имеют естественного представления в виде вектора вещественных чисел. Это позволяет эффективно работать, к примеру, с текстами или даже последовательностями белков. Рассмотрим несколько популярных ядер, которые мы не успели затронуть на занятиях, и их области применения.
\begin{enumerate}
\item \textit{Строковое ядро} \cite{lodhi}. Это ядро часто применяется в биоинформатике, потому что позволяет работать с последовательностями вроде ДНК или белков. При помощи этого ядра замеряется схожесть двух строк.
\par Пусть $ \Sigma $ --- алфавит, а $ s $ и $ t $ --- аргументы ядра. Суть подхода заключается в том, что для каждого слова в алфавите $ \Sigma $ считается его вес в каждом из аргументов, а результатом является сумма произведений таких весов. Чем чаще слово входит в строку как подпоследовательность, тем больше вес. Если между элементами подпоследовательности в слове большое расстояние, то вес уменьшается.

\item \textit{Графовые ядра} \cite{graph}. Аналогично строковому ядру, графовые ядра замеряет схожесть пары графов. При помощи графовых ядер удается работать, например, с молекулами, так как 
\par Одним из графовых ядер является ядро случайного блуждания. Суть этого ядра заключается в том, что сначала в обоих графах осуществляется много случайных блужданий, а затем считается количество путей, которые были получены в обоих графах.

\end{enumerate}

\newpage

\bibliographystyle{unsrt}
\bibliography{reference}

\end{document}
